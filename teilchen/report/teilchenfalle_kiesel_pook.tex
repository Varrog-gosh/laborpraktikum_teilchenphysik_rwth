\documentclass[a4paper,12pt]{article}
\usepackage[utf8]{inputenc}
\usepackage{amssymb,amsmath,uniinput,graphicx}
\usepackage[section]{placeins}
% for uniinput: https://wiki.neo-layout.org/browser/latex/Standard-LaTeX
\usepackage{hyperref}
%\usepackage{multirow}
\usepackage{units}
\usepackage[left=3cm,right=3cm,top=3cm,bottom=3cm]{geometry}
\renewcommand{\familydefault}{\sfdefault}
\setlength{\belowcaptionskip}{6pt}
\hypersetup{pdfinfo = {
	Title={Versuchsprotokoll zur Paulschen Teilchenfalle},
	Author={Knut Kiesel, Tobias Pook},
	Keywords={Teilchenfalle}
}}

\title{Laborpraktikum Teilchenphysik\\ Paulsche Teilchenfalle}
\author{Knut Kiesel\\Tobias Pook}
\date{\today}

\begin{document}
\maketitle
\thispagestyle{empty}
\newpage
\tableofcontents
\setcounter{page}{1}
\newpage

\section{Ziel der Messung} % max 1 Seite
Ziel des Versuches ist die Speicherung von elektrisch geladenen Teilchen und die Bestimmung des Ladungs-Massen Verhältnisses.
Um die Teilchen in einem räumlich begrenzten Feld zu halten, ist ein statisches elektrisches Feld nicht ausreichend, da man damit keine Potentialminima schaffen kann.
Eine Möglichkeit dennoch Teilchen zu fangen ist das Anlegen von phasenverschobenen Wechselspannungen und Gleichspannungen, wobei bei richtiger Einstellungen der Spannungen und Frequenzen die Teilchen stabil in der Falle bleiben.

Für jede räumliche Komponente $i\in\{x,y,z\}$ lautet die Bewegungsgleichung
\begin{align*}\label{mastergleichung}
	\frac{4}{mΩ^2} |\vec{F}_i| + \left( a_i -2q_i \cos\left( 2\xi_i \right) \right) i  + 2k_L \frac{dx}{d\xi_i} + \frac{d^2x}{d\xi_i^2} = B\cos\left( \frac{2ω_W}{Ω}ξ_i \right)
\end{align*}
mit dem gleichstromabhängigen Koeffizienten $a_i = \frac{16KqU_{G,i}}{3Ω^2mr_0^2}$,
dem wechselstromabhängigen Koeffizienten  $q_i = -\frac{4kqU_i}{Ω^2mr_0^2}$,
dem Antribskoeffienzenten $B = \frac{2qU_W}{r_0mΩ^2}$,
dem Luftreibungskoeffizient $k_L = \frac{6πηR}{mΩ}$, der Winkelfrequenz der Dreiphasenspannung $Ω$,
der Winkelfrequenz der zusätzlich an einem Plattenpaar angelegten Wechselspannung $ω_W$
und der normalisierten Zeit $ξ = \frac{Ωt}{2}$.
Die Kraft $\vec{F}$ ist die Gewichtskraft (die nur auf die z-Komponente Auswirkungen hat).
Die Grundschwingung der Lösung wird durch $β_i = \sqrt{a_i + \frac{q_i^2}{2}}$ beschrieben.
Durch Anlegen geeigneter Frequenzen und Spannungen und das Beobachten der Entstehenden Teilchenbewegungen kann mit unterschiedlichen Methoden das Verhältnis von Ladung zur Masse bestimmt werden.


\section{Aufbau und Durchführung}
Die z-Achse verläuft vertikal, die y-Achse ist die Blickrichtung, und die x-Achse liegt senkrecht zu den beiden übrigen.

Die Falle wird aus sechs Kupferringen und 12 Verbindungsstücken zu einem Würfel geklebt.
Nach dem Anlöten und Isolieren der Anschlusskabel wird die Falle mit schwarzem Lack angemalt, um Steulicht in der Kammer zu verringern.
Die Falle wird mittig über der Öffnung für die Spritze an den Anschlusskabeln befestigt, und die Plattform von unten an die Spannungsversorgung angeschlossen (siehe Bild \ref{fallenbild}), welche je nach Hebelstellung die Gleichspannungen oder die zusätzliche Wechselspannung zur Dreiphasenspannung hinzufügt.


\begin{figure}[htb]
		\centering
		\includegraphics[width=0.3\textwidth]{falle.jpg}
		\caption{Versuchsaufbau der Paulschen Teilchenfalle}
		\label{fallenbild}
\end{figure}

Im Spannungsgenerator gibt es mehrere Möglichkeiten die Gleichspannung anzulegen:
Man kann sie auf den beiden gegenüber liegenden Seiten oder zwischen zwei gegenüberliegenden Seiten anbringen.
Die Verschaltung kann man Bild \ref{verschaltung} entnehmen.
\begin{figure}[htb]
		\centering
		\includegraphics[width=0.3\textwidth]{falle.jpg}
		\caption{Verschaltung im Generator}
		\label{verschaltung}
\end{figure}

Aus Sicherheitsgründen wird die Falle durch eine durchsichtige Acrylhaube abgedeckt. Die Haube wurde zusätzlich mit schwarzem Klebeband verkleidet, mit zwei Öffnungen eine für die seitliche Beobachtung der Falle und eine für die von oben angebrachte Lampe.  
Der Versuch wird mit Aluminum Pulver durchgeführt, dieses wird mittels einer Spritze durch eine Öffnung unterhalb der Falle eingebracht. Da zwischen Öffnung und dem stabilen Bereich der Teilchenfalle ein Abstand von ca. 2.5 cm  besteht wurden die Teilchen durch anschnippsen der Spritze in die Falle geschleudert.




\section{Ergebnisse}
Mit einem Messschieber wird der Plattenabstand der Teilchenfalle auf $\unit[(3.05±0.02)]{cm}$ abgeschätzt.
\subsection{Bahnbeschreibung}
Lissajous Figuren
Lissajous Figuren entstehen bei der Überlagerung harmonischer Schwingungen wenn das Verhältnis der Frequenzen rational ist, sich also durch einen ganzzahligen Bruch darstellen lässt. In diesem Fall bildet die Teilchenbahn eine geschlossene
Figur. Die möglichen Formen der Figuren sind sehr vielfältig und hängen vom Frequenzverhältnis und dem Phasenunterschied der Schwingungen ab. 
, Kristallstrukturen, Elipsen: Spannungen nicht gleich\dots
\subsection{Kompensation der Gewichtskraft}
In Gleichung (\ref{mastergleichung}) wird der Einfluss der Luftreibung vernachlässigt und ein Näherungsansatz der Form $z(ξ_z) = Z(ξ_z)+d(ξ_z)$ durchgeführt.
Die z-Komponente wird nun durch
\begin{align*}
	Z(ξ) = Z_0\sin(β_zξ) - \frac{4|\vec{F_z}|}{mβ_zΩ^2}
\end{align*}
beschrieben.
Man sieht, dass die Schwingung um einen konstanten Term verschoben ist, der von $a_z$ und $q_z$ abhängt.
Diese Abhängigkeit besteht nicht mehr, wenn $|\vec{F}_{z}| = |\vec{F_G} + \vec{F_{qE}}| = 0$ ist.

% vorzeichen?
Zu der Dreiphasenspannung wird nun ein zusätzlicher Potentialunterschied zwischen den beiden z-Komponenten angeschlossen, die die Gewichtskraft kompensieren soll.


\subsection{Resonanz}
\subsection{Stabilitätsdiagramm}



\section{Vergleich der Messungen}




% max 30 seiten
\end{document}
