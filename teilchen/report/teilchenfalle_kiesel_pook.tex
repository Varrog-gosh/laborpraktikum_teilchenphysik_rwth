\documentclass[a4paper,12pt]{article}
\usepackage[utf8]{inputenc}
\usepackage{amssymb,amsmath,uniinput,graphicx,placeins}
% for uniinput: https://wiki.neo-layout.org/browser/latex/Standard-LaTeX
\usepackage{hyperref,multirow}
\usepackage[left=3cm,right=3cm,top=3cm,bottom=3cm]{geometry}
\renewcommand{\familydefault}{\sfdefault}
\setlength{\belowcaptionskip}{6pt}
\hypersetup{pdfinfo = {
	Title={Versuchsprotokoll zur Paulschen Teilchenfalle},
	Author={Knut Kiesel, Tobias Pook},
	Keywords={Teilchenfalle}
}}

\title{Laborpraktikum Teilchenphysik\\ Paulsche Teilchenfalle}
\author{Knut Kiesel\\Tobias test Pook}
\date{\today}

\begin{document}
\maketitle
\thispagestyle{empty}
\newpage
\tableofcontents
\setcounter{page}{1}
\newpage

\section{Ziel der Messung} % max 1 Seite
Paulsche Teilchenfalle, da Speicherung im elektrostatischen Feld nicht möglich (Begründen)


\section{Aufbau und Durchführung}
%Bild der Falle
%Schaltplan
%Evakuierung
%Erhebung dar Daten
%Maßstab zum Filmen
%Koordinatensystem
%nomenklaur: eg, xi, \dots


\section{Auswertung}
\section{Diskussion}

\begin{figure}[htb]
		\centering
%		\includegraphics[scale=0.3]{}
%		\caption{\cite{fig:lhc}}
		\label{fig:lhc}
\end{figure}
\FloatBarrier


% Fragen:
% schaltplan richtig? was groß ist widerstand wenn u_g = 0
% wann fängt der versuch an?






% max 30 seiten
\end{document}
