\documentclass[a4paper,12pt]{article}
\usepackage[utf8]{inputenc}
\usepackage{amssymb,amsmath,uniinput,graphicx,hyperref, multirow,siunitx}
\usepackage[section]{placeins}
\usepackage[ngerman]{babel}
\usepackage{feynmp}
\usepackage[left=3cm,right=3cm,top=3cm,bottom=3cm]{geometry}
\renewcommand{\familydefault}{\sfdefault}
\setlength{\belowcaptionskip}{6pt}
\hypersetup{pdfinfo = {
	Title={Versuchsprotokoll zu Produktion und Zerfall von W-Bosonen},
	Author={Knut Kiesel, Tobias Pook},
	Keywords={W-Bosonen}
}}

% for fenyman graphs
\setlength{\unitlength}{\textwidth}
\def\graphheight{0.15}
\def\graphwidth{.4}

% missing transverse energy
\newcommand{\met}{\ensuremath{\not\mathrel{E}}_T}
\graphicspath{{../analyse/}}
\title{Laborpraktikum Teilchenphysik\\ DØ-Experiment: Produktion und Zerfall von W-Bosonen}
\author{Knut Kiesel\\Tobias Pook}
\date{\today}

\begin{document}
\maketitle
\vspace{5cm}
\tableofcontents
\thispagestyle{empty}
\newpage
\setcounter{page}{1}

\section{Ziel des Versuches}
Bei diesem Versuch werden aus Daten des DØ Detektors die Masse und die Zerfallsbreite des W-Bosons
sowie den Wirkungsquerschnitt zur Erzeugung und anschließendem Zerfall in zwei Leptonen bestimmt.

%Dieser Versuch besteht nicht auf einem eigenen Aufbau, sondern es werden Daten des DØ-Experiments
%ausgewertet. Das DØ-Experiment ist neben dem CDF Experiment einer der beiden Hauptdetektoren des
%Tevatron Colliders. Das Tevatron ist ein Proton-Antiproton Beschleuniger, der am Fermilab in
%Illinois, USA bis 2011 betrieben wurde. Der Detektor besteht aus Silizium Trackern, Kalorimetern,
%einer Magnetspule sowie Myonenkammern.

Um den Fund des W Bosons am UA2 Experiment zu bestätigen sowie es genauer zu Untersuchen können auch
beim DØ Experiment verschiedene Zerfallskanäle betrachtet werden. Einer von ihnen ist der Zerfall
in ein Elektron/Positron und ein Neutrino. Bei einer Erzeugung des W Bosons mit Valenzquarks der
Protonen/Antiprotonen ergibt sich der Feynmangraph in erster Ordnung in Abbildung
\ref{fig:feynman}.
\begin{figure}[h]
\centering
\begin{fmffile}{feymanE}
	\begin{fmfgraph*}(\graphwidth,\graphheight)
		\fmfleft{i2,i1}
		\fmfright{o2,o1}
		\fmf{fermion,label=$d$}{i1,v1}
		\fmf{fermion,label=$u$}{v1,i2}
		\fmf{photon,label=$W^-$}{v1,v2}
		\fmf{fermion,label=$ν$}{o1,v2}
		\fmf{fermion,label=$e$}{v2,o2}
	\end{fmfgraph*}
\end{fmffile}
\begin{fmffile}{feymanE+}
	\begin{fmfgraph*}(\graphwidth,\graphheight)
		\fmfleft{i2,i1}
		\fmfright{o2,o1}
		\fmf{fermion,label=$u$}{i1,v1}
		\fmf{fermion,label=$d$}{v1,i2}
		\fmf{photon,label=$W^+$}{v1,v2}
		\fmf{fermion,label=$e$}{o1,v2}
		\fmf{fermion,label=$ν$}{v2,o2}
	\end{fmfgraph*}
\end{fmffile}
\caption{Feynman Diagramm für die Erzeugung von W-Bosonen durch zwei Quarks der ersten Familie und
Vernichtung in zwei Leptonen.}
\label{fig:feynman}
\end{figure}

Den Wirkungsquerschnitt kann man mit
\begin{align*}
	σ = \int_0^1dx_p\int_0^1dx_{\bar{p}} \sum_{i,j} f_i(x_p)f_j(x_{\bar{p}}) \hat{σ}
\end{align*}
berechnen, wobei
\begin{align*}
	\hat{σ} = \frac{1}{N_C}\frac{12π}{m_W^2}\frac{Γ_{qq'}Γ_{eν}}{Γ^2_W}
	\frac{ x_px_{\bar{p}} s Γ_W^2}{\left( x_px_{\bar{p}}s - m_W^2\right)^2 + m_W^2Γ_W^2}
\end{align*}
und
\begin{align*}
	Γ_{ff'} = \frac{N_C}{12} \frac{α}{\sin^2θ_W}m_W
\end{align*}
ist. Dabei sind $f_i$ die Patron-Dichte-Funktionen, $N_C$ die Anzahl an Farben und $\sqrt{s}$ die
Schwerpunktsenergie.
Unter der Annahme, das W kann nur in drei Leptonfamilien und zwei Quarkfamilien zerfallen (das
top-Quark ist zu schwer), kann man über obrige Formel die gesamte Zerfallsbreite des W ausrechnen:
\begin{align*}
	Γ_W = 3Γ_{eν} + 2Γ_{ud}  = \frac{ 3+2N_C}{12} \frac{α}{\sin^2θ_W}m_W
\end{align*}
Damit ergibt sich für
\begin{align*}
	\hat{σ} = \frac{πα²}{12\sin^4θ_W} \frac{ x_px_{\bar{p}} s }{\left( x_px_{\bar{p}}s -
	m_W^2\right)^2 + m_W^4\frac{(3+2N_C)^2}{12^2}\frac{α^2}{\sin^4θ_W}}.
\end{align*}

Über die Masse des W Bosons kann man den elektroschwachen Mischungswinkel, oder auch Weinbergwinkel
$θ_W$ genannt, berechnen:
\begin{align*}
	\cosθ_W = \frac{m_W}{m_Z}
\end{align*}
In dem Versuch werden die Masse des Z Bosons $m_Z$ sowie die Feinstrukturkonstante $α$ als gegeben
vorausgesetzt.

\section{Bestimmung der Masse des W Bosons}
Im Detektor kann die W Masse nicht direkt gemessen werden, sondern nur manche Tochterteilchen. In
dem hier untersuchtem Prozess sind dies ein leichtes geladenes Lepton $e$
\footnote{Elektron wird ab nun als Synonym für Elektron und Positron benutzt} sowie 
ein Neutrino
\footnote{Neutrino wird als Synonym für (Anti-)elektronneutrino benutzt} $ν$. Da die
longitudinale Anfangsbedinungen nicht bekannt sind, werden
nur transversale kinematische Größen betrachtet. Für die Auswertung werden Daten aus dem Kalorimeter
denen des Trackers bevorzugt, da der Kalorimeter die kleinere Unsicherheit hat. So wird zum Beispiel
die transversale Energie eines Elektrons/Positrons
\begin{align*}
	E_{T} = E\sin\left( 2\arctan\left( e^{-\eta} \right) \right)
\end{align*}
aus der Energie im Kalorimeter $E$ und $\eta$ berechnet, wobei $\eta$ sowohl Informationen aus
Tracker und Kalorimeter enthält.
Die Fehlende Transversale Energie wird komplett aus dem Kalorimeter bestimmt und berechnet sich
durch
\begin{align*}
	\met = \sqrt{ {\met}^{2}_{x} + {\met}^{2}_{y}}
\end{align*}
wobei $\met$ als
\begin{align*}
	\met = \left| - \sum \vec{E}_{T} \right|
\end{align*}
definiert ist. Da das Neutrino im Detektor nicht nachgewiesen wird, wird $\met$ als Neutrinoenergie
benutzt. Da nur die transversalen Komponenten von Elektron und Neutrino verwendet werden, kann auch
nur die invariante transversale Masse der Leptonen und damit des W Bosons berechnet werden.
\begin{align*}
	m_T = 2E_T\met\left( 1-\cos(Δ_{e,ν}) \right)\footnotemark
\end{align*}
\footnotetext{Da die beobachteten transversalen Elektronenergien größer als einige $GeV$ sind, wird
im Folgenden die Elektronmasse vernachlässigt.}


\subsection{Beschreibung der simulierten Daten}
Um Untergründe zu entfernen, ist es hilfreich, die gemessenen Daten mit simulierten Ereignissen zu
vergleichen. Da die W Masse im voraus noch nicht bekannt ist, werden Ereignisse für unterschiedliche
W Massen generiert.
Da man den Untergrund durch $W\rightarrow τν\rightarrow eννν$ in Daten schlecht filtern kann, ist
sind auch $τ$ Ereignisse simuliert.

Die Qualität der Simulation ist aber eher durchwachsen, da gewisse Größen wie zum Beispiel der
Abstand des Vertices zum Ursprung, der Polarwinkel $φ$ oder die Elektronisolation sehr schlecht
beschrieben werden.

\subsection{Selektion der Daten}
Da nicht alle aufgezeichneten Ereignisse den zu untersuchendem Prozess zugrunde liegen, müssen die
Daten gefiltert werden, möglichst ohne viele richtige Ereignisse zu verlieren. Es werden nur
Elektronen betrachtet, deren Schauer wie eine elektromagnetische und nicht wie eine hadronische
Kaskade aussieht. Zu dem Schauer muss sich auch in kleinem Abstand eine Spur im Tracker befinden.
Die Elektronen müssen isoliert sein, in $|\eta| < 1.1$ liegen. Es dürfen in dem Ereignis keine
hadronischen Jets mit einem Transversalimpuls von über $\SI{15}{GeV}$ vorkommen und der Primärvertex
darf entlang der Strahlachse nur $\Si{60}{cm}$ vom Mittelpunkt abweichen.

Hierzu werden die
simulierten und gemessenen Daten in Histogrammen übereinandergelegt, und die Unterschiede durch
geeignete Schnitte verringert.

\subsection{Schablonenmethode zur Bestimmung der W Masse}
\subsection{Optimieren der Selektionskriterien}
\subsection{Überprüfung der Ergebnisse durch Untersuchung einer Kontrollvariablen}

\section{Bestimmung des Wirkungsquerschnitts und abgeleiteten Größen}


\section{Vergleich zu anderen Ergebnissen}



%\begin{figure}[htb]
%		\centering
%		\includegraphics[width=\textwidth]{}
%		\caption{}
%		\label{fig:}
%\end{figure}

%\bibliographystyle{plain}
%\bibliography{citebib}{}
\end{document}
