\documentclass[a4paper,12pt]{article}
\usepackage[utf8]{inputenc}
\usepackage{amssymb,amsmath,uniinput,graphicx,hyperref, multirow,siunitx}
\usepackage[section]{placeins}
\usepackage[ngerman]{babel}
\usepackage{feynmp}
\usepackage[left=3cm,right=3cm,top=3cm,bottom=3cm]{geometry}
\renewcommand{\familydefault}{\sfdefault}
\setlength{\belowcaptionskip}{6pt}
\hypersetup{pdfinfo = {
	Title={Versuchsprotokoll zu Produktion und Zerfall von W-Bosonen},
	Author={Knut Kiesel, Tobias Pook},
	Keywords={W-Bosonen}
}}

% for fenyman graphs
\setlength{\unitlength}{\textwidth}
\def\graphheight{0.15}
\def\graphwidth{.4}
\DeclareGraphicsRule{*}{mps}{*}{}

% missing transverse energy
\newcommand{\met}{\ensuremath{\not\mathrel{E}}_T}

% where to find graphics
\graphicspath{{../analyse/}}

\title{Laborpraktikum Teilchenphysik\\ DØ-Experiment: Produktion und Zerfall von W-Bosonen}
\author{Knut Kiesel\\Tobias Pook}
\date{\today}

\begin{document}
\maketitle
\vspace{5cm}
\tableofcontents
\thispagestyle{empty}
\newpage
\setcounter{page}{1}

\section{Ziel des Versuches}
\label{ziel}
Bei diesem Versuch werden aus Daten des DØ Detektors die Masse und die Zerfallsbreite des W-Bosons
sowie den Wirkungsquerschnitt zur Erzeugung und anschließendem Zerfall in zwei Leptonen bestimmt.

%Dieser Versuch besteht nicht auf einem eigenen Aufbau, sondern es werden Daten des DØ-Experiments
%ausgewertet. Das DØ-Experiment ist neben dem CDF Experiment einer der beiden Hauptdetektoren des
%Tevatron Colliders. Das Tevatron ist ein Proton-Antiproton Beschleuniger, der am Fermilab in
%Illinois, USA bis 2011 betrieben wurde. Der Detektor besteht aus Silizium Trackern, Kalorimetern,
%einer Magnetspule sowie Myonenkammern.

Um den Fund des W Bosons am UA2 Experiment zu bestätigen sowie es genauer zu Untersuchen können auch
beim DØ Experiment verschiedene Zerfallskanäle betrachtet werden. Einer von ihnen ist der Zerfall
in ein Elektron/Positron und ein Neutrino. Bei einer Erzeugung des W Bosons mit Valenzquarks der
Protonen/Antiprotonen ergibt sich der Feynmangraph in erster Ordnung in Abbildung
\ref{fig:feynman}.
\begin{figure}[h]
\centering
\begin{fmffile}{feymanE}
	\begin{fmfgraph*}(\graphwidth,\graphheight)
		\fmfleft{i2,i1}
		\fmfright{o2,o1}
		\fmf{fermion,label=$d$}{i1,v1}
		\fmf{fermion,label=$u$}{v1,i2}
		\fmf{photon,label=$W^-$}{v1,v2}
		\fmf{fermion,label=$ν$}{o1,v2}
		\fmf{fermion,label=$e$}{v2,o2}
	\end{fmfgraph*}
\end{fmffile}
\begin{fmffile}{feymanE+}
	\begin{fmfgraph*}(\graphwidth,\graphheight)
		\fmfleft{i2,i1}
		\fmfright{o2,o1}
		\fmf{fermion,label=$u$}{i1,v1}
		\fmf{fermion,label=$d$}{v1,i2}
		\fmf{photon,label=$W^+$}{v1,v2}
		\fmf{fermion,label=$e$}{o1,v2}
		\fmf{fermion,label=$ν$}{v2,o2}
	\end{fmfgraph*}
\end{fmffile}
\caption{Feynman Diagramm für die Erzeugung von W-Bosonen durch zwei Quarks der ersten Familie und
Vernichtung in zwei Leptonen.}
\label{fig:feynman}
\end{figure}

Den Wirkungsquerschnitt kann man mit
\begin{align}
\label{form:xstot}
	σ = \int_0^1dx_p\int_0^1dx_{\bar{p}} \sum_{i,j} f_i(x_p)f_j(x_{\bar{p}}) \hat{σ}
\end{align}
berechnen, wobei
\begin{align*}
	\hat{σ} = \frac{1}{N_C}\frac{12π}{m_W^2}\frac{Γ_{qq'}Γ_{eν}}{Γ^2_W}
	\frac{ x_px_{\bar{p}} s Γ_W^2}{\left( x_px_{\bar{p}}s - m_W^2\right)^2 + m_W^2Γ_W^2}
\end{align*}
und
\begin{align}
	Γ_{ff'} = \frac{N_C}{12} \frac{α}{\sin^2θ_W}m_W
\end{align}
ist. Dabei sind $f_i$ die Patron-Dichte-Funktionen, $N_C$ die Anzahl an Farben und $\sqrt{s}$ die
Schwerpunktsenergie.
Unter der Annahme, das W kann nur in drei Leptonfamilien und zwei Quarkfamilien zerfallen (das
top-Quark ist zu schwer), kann man über obrige Formel die gesamte Zerfallsbreite des W ausrechnen:
\begin{align*}
\label{form:width}
	Γ_W = 3Γ_{eν} + 2Γ_{ud}  = \frac{ 3+2N_C}{12} \frac{α}{\sin^2θ_W}m_W
\end{align*}
Damit ergibt sich für
\begin{align}
\label{form:xscms}
	\hat{σ} = \frac{πα²}{12\sin^4θ_W} \frac{ x_px_{\bar{p}} s }{\left( x_px_{\bar{p}}s -
	m_W^2\right)^2 + m_W^4\frac{(3+2N_C)^2}{12^2}\frac{α^2}{\sin^4θ_W}}.
\end{align}

Über die Masse des W Bosons kann man den elektroschwachen Mischungswinkel, oder auch Weinbergwinkel
$θ_W$ genannt, berechnen:
\begin{align}
	\label{form:wein}
	\cosθ_W = \frac{m_W}{m_Z}
\end{align}
In dem Versuch werden die Masse des Z Bosons $m_Z$ sowie die Feinstrukturkonstante $α$ als gegeben
vorausgesetzt.

\section{Bestimmung der Masse des W Bosons}
Im Detektor kann die W Masse nicht direkt gemessen werden, sondern nur manche Tochterteilchen. In
dem hier untersuchtem Prozess sind dies ein leichtes geladenes Lepton $e$
\footnote{Elektron wird ab nun als Synonym für Elektron und Positron benutzt} sowie 
ein Neutrino
\footnote{Neutrino wird als Synonym für (Anti-)elektronneutrino benutzt} $ν$. Da die
longitudinale Anfangsbedinungen nicht bekannt sind, werden
nur transversale kinematische Größen betrachtet. Für die Auswertung werden Daten aus dem Kalorimeter
denen des Trackers bevorzugt, da der Kalorimeter die kleinere Unsicherheit hat. So wird zum Beispiel
die transversale Energie eines Elektrons/Positrons
\begin{align*}
	E_{T} = E\sin\left( 2\arctan\left( e^{-\eta} \right) \right)
\end{align*}
aus der Energie im Kalorimeter $E$ und $\eta$ berechnet, wobei $\eta$ sowohl Informationen aus
Tracker und Kalorimeter enthält.
Die Fehlende Transversale Energie wird komplett aus dem Kalorimeter bestimmt und berechnet sich
durch
\begin{align*}
	\met = \sqrt{ {\met}^{2}_{x} + {\met}^{2}_{y}}
\end{align*}
wobei $\met$ als
\begin{align*}
	\met = \left| - \sum \vec{E}_{T} \right|
\end{align*}
definiert ist. Da das Neutrino im Detektor nicht nachgewiesen wird, wird $\met$ als Neutrinoenergie
benutzt. Da nur die transversalen Komponenten von Elektron und Neutrino verwendet werden, kann auch
nur die invariante transversale Masse der Leptonen und damit des W Bosons berechnet werden.
\begin{align*}
	m_T = 2E_T\met\left( 1-\cos(Δ_{e,ν}) \right)\footnotemark
\end{align*}
\footnotetext{Da die beobachteten transversalen Elektronenergien größer als einige $GeV$ sind, wird
im Folgenden die Elektronmasse vernachlässigt.}

\subsection{Beschreibung der simulierten Daten}
Um Untergründe zu entfernen, ist es hilfreich, die gemessenen Daten mit simulierten Ereignissen zu
vergleichen. Da die W Masse im voraus noch nicht bekannt ist, werden Ereignisse für unterschiedliche
W Massen generiert.
Da man den Untergrund durch $W\rightarrow τν\rightarrow eννν$ in Daten schlecht filtern kann,
sind auch $τ$ Ereignisse simuliert.

Die Qualität der Simulation ist aber eher durchwachsen, da gewisse Größen wie zum Beispiel der
Abstand des Vertices zum Ursprung, der Polarwinkel $φ$ oder die Elektronisolation sehr schlecht
beschrieben werden.

\subsubsection*{Normierung der simulierten Daten}
Die Simulierten Daten müssen um einen Vergleich mit den gemessenen Daten zu ermöglichen auf die 
integrierte Luminosität der Daten von $\mathcal{L}_{int}=198 \pm \SI{20}{pb^{-1}}$ normiert werden. 
Dazu werden die Daten nach den Vorgaben aus \cite{versuchsanleitung},
in abhängigkeit der ursprünglich generierten Monte Carlo (\textbf{MC}) Ereignissen $N_{gen}$, mit einem Faktor:
\begin{align*}
	\textsl{w}= \frac{\sigma \mathcal{L}_{int}}{N_{gen}} \cdot 0.9
\end{align*}
skaliert. Der Faktor $0.9$ beschreibt einen Korrekturterm, der eine , im Vergleich zum realen Detektor,
zu hoch angenommene Auflösung und Nachweiseffizienz bei der Generierung der MC Ereignisse berücksichtigt.

\subsection{Selektion der Daten}
Da nicht alle aufgezeichneten Ereignisse den zu untersuchendem Prozess zugrunde liegen, müssen die
Daten gefiltert werden, möglichst ohne viele richtige Ereignisse zu verlieren. Es werden nur
Elektronen betrachtet, deren Schauer wie eine elektromagnetische und nicht wie eine hadronische
Kaskade aussieht. Zu dem Schauer muss sich auch in kleinem Abstand eine Spur im Tracker befinden.
Die Elektronen müssen isoliert sein, in $|\eta| < 1.1$ liegen. Es dürfen in dem Ereignis keine
hadronischen Jets mit einem Transversalimpuls von über $\SI{15}{GeV}$ vorkommen und der Primärvertex
darf entlang der Strahlachse nur $\SI{60}{cm}$ vom Mittelpunkt abweichen. Diese Auswahl wird immer
getroffen und nicht gesondert erwähnt wenn angewandt.

Hierzu werden die simulierten und gemessenen Daten in Histogrammen übereinandergelegt
und die Unterschiede durch geeignete Schnitte verringert. Ziel ist vor allem eine gute
Übereinstimmung in $m_T$, da daraus die Masse bestimmt wird. Die Abschätzung von Selektionsgrenzen
wird in Abbildung \ref{fig:abschaetzung} veranschaulicht. Links oben sieht man die transversale
invariante Masse ohne Einschränkungen. Rechts oben sieht man die Verteilung der transversalen
fehlenden Energie. Da Daten und Simulation rechts der gezeichneten Linie gut übereinander stimmen,
wird in Zukunft gefordert, dass $\met > \SI{30}{GeV}$ sein muss. Im Bild links unten sieht man die
Verteilung der transversalen Elektronenergie und die Grenze, ab der Daten und Simulation hier
übereinstimmen. Zuletzt wird rechts unten wieder $m_T$ gezeigt.

\begin{figure}[htb]
	\centering
	\includegraphics[width=.49\textwidth]{mwt1tau.pdf}
	\includegraphics[width=.49\textwidth]{met1tau.pdf}\\
	\includegraphics[width=.49\textwidth]{el_etmet>30tau.pdf}
	\includegraphics[width=.49\textwidth]{mwtmet>30&&el_et>25tau.pdf}
	\caption{Abschätzung der Selektionsgrenzen mit Vergleich von Daten und Simulation. In Blau ist
	der $τ$ Untergrund eingezeichnet. Die Simulation wird hier mit der generierten W Masse von
	$\SI{80.3946}{GeV}$ erzeugt.}
	\label{fig:abschaetzung}
\end{figure}

In Blau sieht man in jedem Bild Ereignisse, in denen das W ein $τ$ und ein $ν_τ$ erzeugt. Das $τ$
zerfällt nach sehr kurzer Zeit wieder in ein $e + ν_e + ν_τ$. Da man die Neutrinos nicht einzeln
Messen kann sondern nur die fehlende Energie, die durch alle Neutrinos zusammen verursacht wird,
kann man diese Ereignisse kaum Filtern, sondern muss den Untergrund über Simulationsvergleiche
abschätzen. Während am Anfang der Anteil der $τ$ Ereignisse noch 8.8\% beträgt, sind es nach beiden
Schnitten 1.4\%.

\subsection{Templatemethode zur Bestimmung der W Masse}
\label{template}
Um die W Masse zu finden, die am Besten zu den Daten passt, werden die für 

\subsection{Bestimmung der Effizienz}
\label{effizienz}
Um einen Vergleich von gemessenen bzw. simulierten Daten mit theoretischen Werten zu ermöglichen muss die Nachweiseffizienz
$\epsilon$ bestimmt werden, die angibt wieviel Prozent aller "wahren" W-Ereignisse nicht selektiert bzw. gemessen wurden.
Es handelt sich bei $\epsilon$ also um das Produkt der, durch geometrische Einschränkungen gegebene, Detektorakzeptanz und der
,durch das Triggersystem und die gewählten Selektionsschnitte bestimmten, Selektionseffizienz. Im hier vorliegenden Fall wird die
Effizienz durch den Quotienten der Anzahl von selektierten und generierten Monte Carlo Ereignissen bestimmt:
\begin{align*}
	\epsilon = \frac{N_{selected}}{N_{gen}} = 0.xx
\end{align*}
Dabei werden hier die MC-Sample natürlich nicht auf die Luminosität normiert bevor die Schnitte, zur Bestimmung von $N_{selected}$, durchgeführt
werden.   
Es ist dabei zu beachten, dass die so berechnete Effizienz davon ausgeht das die Nachweiseffizienz für simulierte und gemessene
Daten gleich ist. Man geht also davon aus das die Simulation die Detektoreigenschaften perfekt abbildet. Die Unterschiede in der
Nachweiseffizienz zwischen simulation und Daten werden im Folgenden,nach Versuchsanleitung \cite{versuchsanleitung} durch einen 
Korrekturfaktor $0.9\pm0.1$ abgeschätzt (Es handelt sich dabei um den gleichen Faktor der auch schon bei der Normierung der simulierten Daten 
verwendet wurde), wobei der Fehler auf diesen Wert diverse systematische Unsicherheiten der Simulation
beeinhalten. Diese Unsicherheiten wurden in \cite{versuchsanleitung} leider nicht weitergehenend erklärt, eine genauere Beschreibung
ist auf Grund fehlender Daten nicht möglich. Im Fall einer realen Analyse würde man es wohl ohnehin vorziehen die Nachweiseffizienz aus
den aufgezeichneten Daten zu bestimmen. Hierfür werden meisstens andere, besser vermessene, Resonanzen des totalen Wirkungsquerschnitt genutzt 
in denen auch Elekronen und Neutrinos entstehen. Dann lässt sich z.B. mit der "Hit & Miss" Methode die Nachweiseffizienz bestimmen. Auf Grund
der stark vorselektierten Daten-Sample war dies für diesen Versuch nicht möglich.
\subsection{Optimieren der Selektionskriterien}
\subsection{Überprüfung der Ergebnisse durch Untersuchung einer Kontrollvariablen}

\section{Bestimmung des Wirkungsquerschnitts und abgeleiteten Größen}
Der Wirkungsquerschnitt einer Reaktion lässt sich theoretisch aus der Anzahl gemessener Ereignisse und der 
integriereten Luninosität durch $N_{obs}=\sigma \cdot \mathcal{L}_{int}$ berechnen. Im realen Experiment
muss die Anzahl selektierter Ereignisse noch mit der Nachweiseffizienz $\epsilon$ bzw. hier $\epsilon \cdot corr$ 
korregiert werden (siehe \ref{effizienz}).hierbei wird nur $corr$ als fehlerbehaftet angenommen, $\epsilon$ dagegen wegen der hohen Statistik 
als fehlerfrei. Der Wirkungsquerschnitt für die Reaktion $q+\bar{q}\rightarrow W \rightarrow e+ \nu$ wurde bestimmt zu :
\begin{align*}
	\sigma = \frac{N_{selected}}{\epsilon \cdot corr \cdot \mathcal{L}_{int}} = xx \pm \SI{ex}{nb}
\end{align*}
Der berechnete Wert weicht $xx\sigma$ (hier Standardabweichungen) vom Literaturwert $\sigma_{theo}=2.58 \pm \SI{0.09}{nb}$
\cite{versuchsanleitung} ab und ist somit innerhalb von $ 3\sigma$ mit diesem kompatibel.
\subsection{Bestimmung des Farbfaktors}
Der im letzten Abschnitt benutzte theoretische Wert für den Wirkungsquerschnitt geht von einem Farbfaktor von $N_{\textsl{C}}=3$ aus, die
Messung ist mit diesem Wert für $N_{\textsl{C}}$ also kompatibel. Die angegebenen Formeln erlauben es leider nicht die Änderung des theoretischen 
Wirkungsquerschnitt zu berechnen. Dazu müsste das Integral in \ref{form:xstot} mit passenden  Partondichtefunktionen für verschiedene Werte
von $N_{\textsl{C}}$ in \ref{form:xscms} integriert werden. Selbst diese Lösung wäre nur in führender Ordnung und wohl nicht geeignet um Vergleiche
zu den Messwerten durchzuführen.Der Wirkungsquerschnitt hängt nicht auf triviale Weise mit dem Farbfaktor zusammen, deshalb könnten wohl auch nicht die 
NLO Korrekturwerte für $N_{\textsl{C}}=3$ benutzt werden.
 Ein berechnen mit Software von theoretikern scheitert daran , dass die Vollidioten im Jahr 2012 immernoch mit
Fortranprogammen rum hantieren!
\subsection{Bestimmung des Weinbergwinkel}
Der Weinbergwinkel $\theta_{W}$ beschreibt den elektroschwachen Mischungswinkel und beschreibt wie die
,vor der Symmetriebrechung durch den Higgsmechanismus masselosen,Felder der elektroschwachen Eichbosonen nach der Symmetrieberechung 
mischen. Er ist, wie in \ref{ziel} Formel \ref{form:wein} bereits erwähnt, direkt mit dem Masseverhältnis der massiven Eichbosonen W und Z verknüpft.
Mit der aus \cite{versuchsanleitung} entnommenen Z-masse $m_{Z}=91.227 \pm \SI{0.041}{GeV}$ und der in \ref{template} bestimmten W Masse ergibt
 sich fürden Weinbergwinkel:
\begin{align*}
	sin(\theta_{W})^{2} = 1 - cos(\theta_{W})^{2} = 1 - (\frac{m_{W}}{m_{Z}})^{2} = xx \pm \SI{ex}{nb}
\end{align*}
Die systematischen und statistischen Fehler auf $m_{w}$ werden zur Bestimmung des Fehler auf $sin(\theta_{W})^{2}$ quadratisch addiert.
Für einen Vergleich mit dem aktuellen Weltmittelwert muss im Ergebnis noch berücksichtigt werden das es sich bei \ref{form:wein} um eine
Näherung erster Ordnung handelt. Die genaueren Berechnungen zur Bestimmung des Weltmittlewert berücksichtigen noch Strahlugskorrekturen aus höheren
Ordnungen, dieser Unterschied wird gemäß der Beschreibung in \cite{versuchsanleitung} durch einen Korrekturfaktor von $corr_{strahlung}=1.06$
berücksichtigt. Der Vergleich von korrigiertem und Weltmittelwert 
\begin{align*}
	sin(\theta_{W})_{korrigiert}^{2} = xx \pm \SI{ex}{nb} \vspace{1cm} sin(\theta_{W,Welt})^{2} = xx \pm \SI{ex}{nb}
\end{align*}
zeigt, dass der berechnete und korrigierte Weinbergwinkel mit einer Abweichung von $xx\sigma$ vom Literaturwert mit diesem 
innerhalb von $3\sigma$ kompatibel ist.Die Messung kann also den bisherigen Messwert innerhalb des gewählten statistischen 
Toleranzbereich bestätigen.
\subsection{Bestimmung der W-Zerfallsbreite}
Die Zerfallsbreite des W-Boson wird mit Hilfe von Formel \ref{form:width} bestimmt. Für die konkrete berechnung
 wurden $\alpha = \frac{1}{128}$ und $N_{\textsl{C}}=3$ verwendet und als fehlerfrei angenmommen, sin(\theta_{W})^{2} 
 wird in \ref{form:width} durch $corr_{strahlung}\cdot(1-(frac{mw}{mz})^{2}$ ersetzt und die statistischen und 
 systematischen Fehler auf m_{W} und m_{Z} werden als unabhängige Fehlerquellen für das Ergebnis fortgepflanzt. Für 
 einen Vergleich mit den Literaturwerten muss die gemessene W-Breite nach \cite{versuchsanleitung} mit einem Korrekturfaktor
 von $corr_{HO}=1.09 \pm 0.01$ multipliziert werden um das fehlen höherer Ordnungen in der Rechnung zu berücksichtigen, der
 Fehler auf diesen Korrekturfaktor wurde bei der Fehlerfortpflanzung ebenfalls als zusätzliche systematisch Fehlerquelle
 berücksichtigt.Die Rechnung ergibt für die korrigierte W-Zerfallsbreite:
 \begin{align*}
	Γ_{W,korrigiert} = xx \pm \SI{ex}{GeV} 
\end{align*}
Dieser Wert weicht $xx\sigma$ von dem in \cite{versuchsanleitung} angegebenen Literaturwert von $Γ_{W,theo} = xx \pm \SI{10}{GeV}$
ab und ist somit innerhalb von $3\sigma$ mit diesem vereinbar.
\section{Diskussion der systematischen Unsicherheiten}
In diesem Abschnitt werden die systematischen Unsicherheiten bei der W Massenbestimmung an Hadronkollidern am beispiel vom Tevatron diskutiert 
und mit denenan Elektron-Positron Kollidern am Beispiel von LEP verglichen. Da der Anfangszusand bei $e^{+}+e^{-}$ Kollisionen neutral ist können einzelne W-Bosonen 
nicht in führender Ordnung erzeugt werden. Deshalb wurde im LEP Experiment die W Paarproduktion zur Massenbestimmung genutzt \cite{Achard:2005qy}
und hierbei die Zerfallskanäle $W+W \rightarrow qql\nu$ und $W+W \rightarrow qqqq\nu$ genutzt.  
Die benutzten Angaben beziehen sich ,soweit nicht anders angegeben, auf die Analysen in 
\cite{Abachi:1996ey} für Hadronkollider und \cite{Achard:2005qy} für Elektron-Positron Kollider. Zunächst wird der Einfluss von Unsicherheiten
berachtet die beide Kollidertypen betrifft:

\begin{itemize}
	\item \textbf{Luminosität:} \\
	Die integrierte Luminosität geht in beiden Experimenten als systematische Unsicherheit bei der Bestimmung der Masse des W Bosons ein
	, die Genauigkeit mir der diese Größe bestimmt werden kann ist allerdings sehr unterschiedlich. Beim DØ Experiment wird die Luminosität
	durch Messungen des Wirkungsquerschnitt der inklusiven tief inelastischen Proton-Antiproton bestimmt \cite{2011arXiv1106.5182P}.Dazu
	wird ein System von Plastikszintillatoren im Winkelbereich $2.7 \leq |\eta| \leq 4.4$ verwendet. Die Genauigkeit dieses Nachweisverfahren wurde
	im laufe der Betriebszeit von Werten $ > 10%$ bis auf $ 6.1%$ verbessert. Bei LEP konnte die Luminosität mit einer Unsicherheit von weniger
	als $0.1%$ bestimmt werden, dazu wird der Wirkungsquerschnitt für die Bhabba-Streuung mit kleinem Winkel gemessen. Dieser QED-Prozess
	wird durch den t-Kanal bestimmt und lässt sich theoretisch sehr genau berechnen.
	\item \textbf{Luminosität:} \\
	
\end{itemize}

\bibliographystyle{plain}
\bibliography{citebib}{}
\end{document}
